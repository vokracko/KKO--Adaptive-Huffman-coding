\documentclass[12pt,a4paper,titlepage,final]{article}
\usepackage[czech]{babel}
\usepackage[utf8]{inputenc}
\usepackage[bookmarksopen,colorlinks,plainpages=false,urlcolor=blue,unicode]{hyperref}
\usepackage{url}
\usepackage{float}
\usepackage{ifthen}
\usepackage[dvipdf]{graphicx}
\usepackage[top=3.5cm, left=2.5cm, text={17cm, 24cm}, ignorefoot]{geometry}

\begin{document}
\newpage

%%%%%%%%%%%%%%%%%%%%%%%%%%%%%%%%%%%%%%%%%%%%%%%%%%%%%%%%%%%%%%%%%%%%%%%%%%%%%%
% titulní strana
\def\name{Lukáš Vokráčko}
\def\login{xvokra00}
\def\subject{Kódování a komprese dat}
\def\project{Adaptivní Huffmanovo kódování}

\newboolean{frontpage}
% \setboolean{frontpage}{true}
\setboolean{frontpage}{false}

\ifthenelse{\boolean{frontpage}}
{
	\pagestyle{empty}
	\input{title.tex}
	\tableofcontents
	\newpage
	\pagestyle{plain}
}
{
	\pagestyle{plain}
	\vspace*{5px}
	\hfill \name, \login \\
	\vspace*{5px}
	{\LARGE \subject}  \\
	{\LARGE \project}  \\
}

\pagenumbering{arabic}
\setcounter{page}{1}
%%%%%%%%%%%%%%%%%%%%%%%%%%%%%%%%%%%%%%%%%%%%%%%%%%%%%%%%%%%%%%%%%%%%%%%%%%%%%%

\section{Huffmanovo kódování}
Huffmanovo kódování je bezeztrátové prefixové kódování s proměnnou délkou kódu, 
které přiřazuje kódy znakům ze vstupu dle jejich četnosti. Znakům s vysokou četností
jsou přiřazeny kódy s menším bitovou délkou a méně častým znakům pak kódy s větší bitovou délkou. 
Tím je docíleno komprese v případě, že na vstupu je pouze zlomek z celé vstupní abecedy
nebo je četnost jednotlivých znaků jiná než rovnoměrná.

Kód každého znaku je odvozen dle pozice v Huffmanově stromu. Kód znaku je sestrojen průchodem Huffmanovým stromem
od kořene k listovému uzlu, jež odpovídá danému znaku.
Pokud je při průchodu zvolena cesta levým podstromem, je ke kódu přidán bit $0$, v případě průchodu
pravým stromem je to bit $1$.

\section{Adaptivní Huffmanovo kódování}
Adaptivní verze Huffmanova kódování spočívá v kódování znaků za letu, kdy program dopředu nezná četnosti
znaků a kódy jednotlivých znaků se během zpracování vstupu mění. To přináší problém modifikace stromu,
kde je nutné při odeslání každého znaku inkrementovat počítadlo četnosti a přestavět strom tak, aby byl zachovával vlastnosti 
Huffmanova stromu, což znamená, že spojeny jsou spolu vždy 2 uzly stromu s nejmenší četností.

	\subsection{Tvorba Huffmanova stromu}
	Na počátku je strom tvořen pouze uzlem odpovídajícím symbolu \texttt{NYT} (not yet transmitted) a číslem uzlu rovno 1.
	Uzel \texttt{NYT} se ve stromu nachází po celou dobu, jeho četnost je pevně nastavena na nulu a jeho pozice
	se během tvorby stromu mění. Tento uzel slouží jako oddělovač symbolů, které jsou přenášeny poprvé, tedy ve své nezakódované podobě.

		\subsubsection{Vkládání znaku}
		Při vložení nového znaku dojde k vytvoření dvou uzlů v místě, kde je uložen symbol \texttt{NYT}. 
		Pravý uzel bude mít číslo uzlu o jedna větší než rodičovský uzel a v levém uzlu bude toto číslo větší o dva.
		Levý uzel je uzlem nově reprezentujícím \texttt{NYT} a do pravého uzlu je vložen nový symbol.
		Po vložení znaku je provedena modifikace stromu popsaná níže, kde jako počáteční uzel je zvolen uzel reprezentující nový symbol.

		\subsubsection{Modifikace stromu}
		Postup při modifikaci stromu je následující:
			\begin{enumerate}
				\item Je vyhledán uzel, který má počítadlo četností nastavené na stejné číslo jako aktuální uzel a zároveň má nižší číslo uzlu.
				\item Existuje-li takový uzel a zároveň to není přímý rodič aktuálního uzlu, dojde k záměně těchto uzlů a to až na číslo uzlu.
				\item Je inkrementováno počítadlo četností v aktuálním uzlu a jako aktuální uzel je nastaven rodič aktuálního uzlu uzlu.
				\item Je li aktuální uzel kořenem stromu je proces modifikace ukončen. V opačném případě se pokračuje bodem 1.
			\end{enumerate}

	\subsection{Kodér}
	Funkce kodéru spočívá ve čtení vstupního proudu dat a zápisu kódovaných symbolů do výstupního proudu dat.
	Pokud kodér přijme znak, který ještě není uložen v Huffmanově stromu (není pro něj vytvořený kód),
	kodér odešle kód pro symbol \texttt{NYT}, což značí, že bude přenášen nezakódovaný symbol.
	Poté odešle znak v nezakódované podobě a znak vloží do Huffmanova stromu.
	Pokud kodér přečte na vstupu znak, jež se ve stromu již vyskytuje, pouze odešle kód daného znaku, inkrementuje počítadlo
	výskytů a modifikuje strom, kde aktuální uzel bude odpovídat uzlu reprezentují daný znak.
	V případě konce souboru je nutné tuto událost zakódovat do výstupu, jinak by mohlo dojít k neplatnému stavu na 
	straně dekodéru, neboť poslední symbol může být zakódován na pomezí dvou bytů a dekodér by pokračoval v dekódování zbylých bitů.
	Pro tuto situaci se využije opět symbolu \texttt{NYT}.

	\subsection{Dekodér}
	Dekodér čte jednotlivé bity ze vstupního proudu dat a prochází Huffmanovým stromem dle hodnoty přečteného bitu až do listového uzlu.
	V listovém uzlu je pak přečten znak, jež odpovídá danému kódu a ten jen vypsán do výstupního proudu 
	v případě, že se jedná o symbol jiný než \texttt{NYT}. 
	V případě, že je přečten \texttt{NYT} nebo jde o první symbol, je tento symbol vložen do Huffmanova stromu.

\section{Možné problémy}
Jeden z možných problémů je přetečení čítače, ke kterému by došlo v případě, kdy uzel dosáhne četnosti $2^{64}$.
Tento krajní případ není v tomto projektu ošetřen, neboť takový případ považuji za silně nepravděpodobný vzhledem k tomu,
jakou velikost souboru toto číslo představuje (milióny TB). V případě, že by existoval požadavek na řešení tohoto problému,
tak je možné použít následující postup: Četnost každého symbolu je snížena na polovinu a strom je přestavěn tak, aby
zachovával vlastnosti Huffmanova stromu.

\end{document}
