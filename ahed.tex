\documentclass[12pt,a4paper,titlepage,final]{article}
\usepackage[czech]{babel}
\usepackage[utf8]{inputenc}
\usepackage[bookmarksopen,colorlinks,plainpages=false,urlcolor=blue,unicode]{hyperref}
\usepackage{url}
\usepackage{float}
\usepackage{ifthen}
\usepackage[dvipdf]{graphicx}
\usepackage[top=3.5cm, left=2.5cm, text={17cm, 24cm}, ignorefoot]{geometry}

\begin{document}
\newpage

%%%%%%%%%%%%%%%%%%%%%%%%%%%%%%%%%%%%%%%%%%%%%%%%%%%%%%%%%%%%%%%%%%%%%%%%%%%%%%
% titulní strana
\def\name{Lukáš Vokráčko}
\def\login{xvokra00}
\def\subject{Kódování a komprese dat}
\def\project{Adaptivní Huffmanovo kódování}

\newboolean{frontpage}
% \setboolean{frontpage}{true}
\setboolean{frontpage}{false}

\ifthenelse{\boolean{frontpage}}
{
	\pagestyle{empty}
	\input{title.tex}
	\tableofcontents
	\newpage
	\pagestyle{plain}
}
{
	\pagestyle{plain}
	\vspace*{5px}
	\hfill \name, \login \\
	\vspace*{5px}
	{\LARGE \subject}  \\
	{\LARGE \project}  \\
}

\pagenumbering{arabic}
\setcounter{page}{1}
%%%%%%%%%%%%%%%%%%%%%%%%%%%%%%%%%%%%%%%%%%%%%%%%%%%%%%%%%%%%%%%%%%%%%%%%%%%%%%

\section{Úvod}
	Algoritmus dosahuje konstantní paměťové složitosti dosahující maxima v závislosti na počtu reálně počtu reálně
	použitých symbolů, které jsou přenášeny a to díky tomu, že pro kažďý symbol je nutné vytvořit dva uzly ve stromu.
	
	Pro snadné vyhledání listových uzlů pro jednotlivé znaky je použito i pomocné pole ukazatelů,
	které je ve své první polovině indexováno možnými znaky a ve druhé polovině jsou uloženy ukazatele
	na nelistové uzly stromu.
	Průchodem tohoto pole je efektivně řešeno vyhledávání ulzů se stejnou vahou a co nejmenenším číslem.

	Při kódování znaků je nejdříve jištěn listový uzel odpovídající danému znaku a poté je sestrojen
	jeho kód průchodem stromu směrem ke kořeni. Po narazení na kořen je nutno tento kód reverzovat aby odpovídal reálnému 
	huffmanově kódu.

	Pro vytvoření projektu bylo vytvořeno řešení v čistém jazyce C
\subsection{Kódování}


\subsection{Dekódování}

	Při dekódování jsou ze vstupního souboru čteny jednotlivé bity
	a dle jejich hodnoty se prochází stromem od kořene až do listového uzlu.
	V listovém uzlu je pak přečten znak, jež odpovídá danému kódu a ten jen vypsán do výstupního souboru
	v případě že se jedná o symbol jiný než NYT. V dalším průchodu se opět začíná v kořeni stromu.
	V případě že je přečten NYT nebo jde o první symbol, je tento symbol vložen do stromu na místo
	kde se do této doby vyskytoval NYT, kde jsou vytvořeny 2 listy, 1 pro symbol a druhý pro NYT.
	Poté je provedena kontrola stromu, zda je stále platným Huffmanovým stromem a případně jeho modifikace.

\subsubsection{Problémy}
	Znak, který je přenášen poprvé a tudíž není zakódován se může vyskytovat na přelomu více bytů a 
	je tedy nutné ukládat přečtený bity do pomocného bufferu.

	Aby bylo možné poznat, kde končí kód, jež díky pohyblivé délce kódu nemusí být na stejném místě jako je přečtený EOF
	tak je použit opět NYT. Konec souboru je tak detekován nastavbením přečtení NYT jako posledního symbolu
	a konce vstupu ze souboru.

	kód málo používaného znaku může dosahovat větší délky než je jeden byte a tudíž je při generování kódu nutné
	používat pomocný buffer, který se stará o uložení kódu v celé své délce

	Další z možných problémů je přetečení čítače, ke kterému by došlu v případě, že uzel by dosáhl váhy $2^{64}$.
	Tento krajní případ není v tomto projektu oštřen, nebot takový případ považuji za silně nepravděpodobný vzhledem k tomu,
	jak veliký souboru toto číslo představuje (milióny terabytů). V případě že by se tento problém chtěl řešit tak je možné použít následující
	řešení

\subsection{Modifikace stromu}
Modifikace stromu probíhá následujícím způsobem:
když je přečten znak, pokusím se najít uzel stromu, který má stejnou váhu jako uzel reprezentující symbol ale má nižší číslo a závislostiroveň 
není přímým rodičem symbolu. Pokud takový uzel najdu, tak zaměním tyto dva uzly i s jejich podstromy ale s tím, že čísla těchto dvou uzlů zůstanou 
na stejných místech. Tento postup se opakuje pro každý rodičovský uzel až ke kořeni stromu.

Najití odpovídajícího uzlu je provedeno v konstantním čase, neboť všechny uzly ustromu jsou uloženy v pomocném poly.
Najítí se tak zjednodušeno na hledání minimálního prvku v poli.

určité optimalizace by bylo možné dosáhnout ukládáním uzlů se stejnou vahou do samostatné struktury řazené dle pořadového čisla
která by držela ukazatele a nebylo by nutné procházet všechny položky pole ukazatelů na uzly stromu.
Na jednu stranu by se tím zrychlilo vyhledání odpovídajícího uzlu, ale přibyla by řežie pro řazení a udržování těchto struktur aktuálních.


\end{document}
